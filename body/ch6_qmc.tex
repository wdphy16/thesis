\chapter{Unbiased evaluation of ground state}

The previous part have let us seen exact sampling methods, particularly autoregressive neural networks (ARNN), achieve higher accuracy and efficiency in approximating classical many-body systems. Now we move on to investigate quantum systems, which are known to be more intricate than their classical counterparts. Even though we only focus on their ground states, rather than finite-temperature ensembles or dynamics, they involve peculiar issues that do not exist in the classical world, such as vanishing energy gaps, sign structures, and entanglements.

In this part, we first review the traditional methods of exact diagonalization (ED), path integral Monte Carlo, variational Monte Carlo (VMC), and tensor networks. They not only share some common issues with Markov chain Monte Carlo (MCMC) and variational methods discussed in the previous part, but are also affected by the peculiar issues in quantum systems. Then we introduce a variational ansatz named tensor-RNN, which combines the strengths of tensor networks and recurrent neural networks (RNN) to shed new light on both the analytical and the numerical aspects of quantum systems. Moreover, we discuss the results from VarBench, an extensive project to benchmark the performances of variational methods on quantum many-body systems, which witnesses the recent development in the field, including exact sampling methods. Lastly, we present NetKet, a software package that we have developed to ease the implementation of all the computational techniques aforementioned, with attention on the integration of exact sampling in it.

\section{Imaginary time evolution (ITE)}

The ground state vector $\ket{\psi_0}$ provides the most complete information of a quantum many-body system at ground state, which allows obtaining all its properties. We start from a straightforward analytical method to evaluate $\ket{\psi_0}$, using the basic properties of eigenstates. We apply the exponential of the system's Hamiltonian $\hat{H}$, with a scaling parameter $\tau$, on an arbitrary initial state $\ket{\psiinit}$, and obtain
\begin{equation}
\ket{\psi_\tau} = \rme^{-\tau \hat{H}} \ket{\psiinit},
\label{eq:ite-psi-tau}
\end{equation}
where we temporarily ignore the normalization of the states. Because of the orthogonality and the completeness of the eigenstates, the initial state can be decomposed into a linear combination of them:
\begin{equation}
\ket{\psiinit} = \sum_i c_i \ket{\psi_i},
\end{equation}
where $\ket{\psi_i}$ is the $i$-th lowest eigenstate, and $c_i$ is the corresponding coefficient that can be determined by projection. Under the action of the Hamiltonian, each eigenstate is scaled by its energy $E_i$:
\begin{equation}
\hat{H} \ket{\psi_i} = E_i \ket{\psi_i}.
\end{equation}
The exponential of an operator is defined by the Taylor expansion:
\begin{equation}
\rme^{-\tau \hat{H}} = \sum_{j = 0}^\infty \frac{1}{j!} (-\tau \hat{H})^j,
\label{eq:ite-taylor}
\end{equation}
assuming it converges. Therefore, \cref{eq:ite-psi-tau} becomes
\begin{equation}
\ket{\psi_\tau} = \sum_i c_i \rme^{-\tau E_i} \ket{\psi_i}.
\end{equation}
In the limit $\tau \to \infty$, as the exponential $\rme^{-\tau E_0}$ grows faster than all other terms, the resulting state $\ket{\psi_\tau}$ is dominated by the ground state $\ket{\psi_0}$. This is true regardless of the coefficients $c_i$, as long as $c_0 \neq 0$, i.e., the initial state is not orthogonal to the ground state, which is usually fulfilled if the initial state is randomly chosen. In the case of degenerate ground states, this method produces an arbitrary one among them, depending on the choice of the initial state.

Recovering the normalization, we have
\begin{equation}
\ket{\psi_0} = \lim_{\tau \to \infty} \frac{\ket{\psi_\tau}}{\sqrt{\ip{\psi_\tau}}}.
\label{eq:ite}
\end{equation}
This method is known as the imaginary time evolution (ITE)~\cite{goldberg1967integration}, because it has a similar form to the conventional notation of time evolution under the Schrödinger equation:
\begin{equation}
\ket{\psi(t)} = \rme^{-\rmi \hat{H} t} \ket{\psiinit},
\end{equation}
which becomes \cref{eq:ite-psi-tau} if we let the time be an imaginary number $t = -\rmi \tau$.

The asymptotic convergence speed of the limit in \cref{eq:ite} depends on the energy gap $\Delta E = E_1 - E_0$ between the ground state $\ket{\psi_0}$ and the first excited state $\ket{\psi_1}$, as we can see from the ratio of their overlaps with $\ket{\psi_\tau}$:
\begin{equation}
\frac{\ip{\psi_1}{\psi_\tau}}{\ip{\psi_0}{\psi_\tau}} \propto \rme^{-\tau \Delta E}. \label{eq:ite-converge}
\end{equation}
This is a root cause of various numerical difficulties in gapless systems, such as spin liquids and systems around critical points.

\section{Exact diagonalization (ED)}

When numerically implementing ITE, we discretize the limit $\tau \to \infty$ into an iterative scheme, and do the normalization in each iteration to ensure numerical stability:
\begin{align}
\ket{\psi'_k} &= \rme^{-\Delta \tau \hat{H}} \ket{\psi_k}, \\
\ket{\psi_{k + 1}} &= \frac{\ket{\psi'_k}}{\sqrt{\ip{\psi'_k}}}.
\end{align}
Meanwhile, we can only keep the first order in the Taylor expansion of $\rme^{-\Delta \tau \hat{H}}$, therefore \cref{eq:ite-taylor} becomes
\begin{equation}
\rme^{-\Delta \tau \hat{H}} \approx 1 - \Delta \tau \hat{H}.
\end{equation}
Given that the time step $\Delta \tau$ is small enough and the number of iterations $k$ is large enough, $\ket{\psi_k}$ converges to the desired $\ket{\psi_0}$. This method is known as the power iteration~\cite{mises1929praktische}. It falls in a broader family of iterative methods to compute the lowest one or few eigenvalues and eigenvectors of a matrix, where the Lanczos algorithm~\cite{lanczos1950iteration} is particularly well-known, following various algorithmic improvements and software implementations~\cite{knyazev2001toward, stathopoulos2010primme}. The convergence speeds of those algorithms also depend on the energy gap $\Delta E$, as shown in \cref{eq:ite-converge}.

In the context of quantum physics, this kind of methods are referred to as the exact diagonalization (ED)~\cite{weisse2008exact}, because they do not involve any approximation or stochastic estimation in the obtained states. However, we can only perform ED on small systems, because the storage of the entire state vector and the multiplication with the Hamiltonian become impractical on larger systems due to the exponentially high dimension of the Hilbert space, as discussed in \cref{sec:qu-sys}. As of this writing, the largest ED computation has been performed on $50$ spin-$1/2$ particles~\cite{wietek2018sublattice}, which utilizes the symmetries of the physical system, as well as sophisticated parallelization and storage strategies on the supercomputer. Despite its exponential computational complexity, ED is still a necessary method to provide initial insights at small system sizes when investigating an unfamiliar quantum model, and it serves as the ground truth to test the correctness of more advanced computational techniques.

\section{Quantum Monte Carlo (QMC)}

On larger systems where the exact evaluation of the ground state is impractical, we can use stochastic methods to approximate it, which fit into the framework of Monte Carlo estimation discussed in \cref{sec:monte-carlo}. Quantum Monte Carlo (QMC) has become a gross term referring to these methods.

In the following, we present a QMC scheme directly following the above derivation of ITE. Temporarily ignoring the normalization, we discretize each component of the ground state vector $\ip{\vs}{\psi_0}$ with
\begin{align}
\ip{\vs}{\psi_0} &= \lim_{\tau \to \infty} \mel{\vs}{\rme^{-\tau \hat{H}}}{\psiinit} \\
&= \lim_{N_\tau \to \infty} \mel{\vs}{\edthp^{N_\tau}}{\psiinit} \\
&= \lim_{N_\tau \to \infty} \sum_{\vs_1, \ldots, \vs_{N_\tau}}\!\!\!\!\mel{\vs}{\edth}{\vs_1} \mel{\vs_1}{\edth}{\vs_2} \cdots \mel{\vs_{N_\tau - 1}}{\edth}{\vs_{N_\tau}} \ip{\vs_{N_\tau}}{\psiinit},
\end{align}
where the auxiliary configurations $\vs_1, \ldots, \vs_{N_\tau}$ are inserted using the completeness of basis states. They are used to define the short-time propagator
\begin{equation}
G(\vs, \vs') = \mel{\vs}{\edth}{\vs'}.
\end{equation}
As long as $\Delta \tau$ is small, $G(\vs, \vs')$ can be computed with high accuracy, where the Trotter--Suzuki decomposition~\cite{suzuki1976generalized} is usually applied.

Using this discretization to estimate the energy, and recovering the normalization, we have
\begin{align}
E &= \lim_{\tau \to \infty} \frac{\ev{\rme^{-\tau \hat{H}} \hat{H}\,\rme^{-\tau \hat{H}}}{\psiinit}}{\ev{\rme^{-\tau \hat{H}} \rme^{-\tau \hat{H}}}{\psiinit}} \\
&= \lim_{\tau \to \infty} \frac
{\ev{\rme^{-2 \tau \hat{H}} \hat{H}}{\psiinit}}
{\ev{\rme^{-2 \tau \hat{H}}}{\psiinit}} \label{eq:pimc-commute} \\
&= \lim_{\tau \to \infty} \frac
{\sum_{\vs, \vs'} \mel{\psiinit}{\rme^{-2 \tau \hat{H}}}{\vs} \mel{\vs}{\hat{H}}{\vs'} \ip{\vs'}{\psiinit}}
{\ev{\rme^{-2 \tau \hat{H}}}{\psiinit}} \\
&= \lim_{\tau \to \infty} \frac
{\sum_\vs \mel{\psiinit}{\rme^{-2 \tau \hat{H}}}{\vs} \ip{\vs}{\psiinit} \sum_{\vs'} \mel{\vs}{\hat{H}}{\vs'} \frac{\ip{\vs'}{\psiinit}}{\ip{\vs}{\psiinit}}}
{\ev{\rme^{-2 \tau \hat{H}}}{\psiinit}} \\
&= \lim_{N_\tau \to \infty} \sum_\svs \varPi(\svs) E_\text{loc}(\vs_{N_\tau}), \label{eq:pimc}
\end{align}
where
\begin{align}
\svs &= \{\vs_0, \vs_1, \ldots, \vs_{N_\tau}\}, \\
\varPi(\svs) &= \frac{\tilde{\varPi}(\svs)}{\sum_\svs' \tilde{\varPi}(\svs')}, \\
\tilde{\varPi}(\svs) &= \psiinit^*(\vs_0) G(\vs_0, \vs_1) G(\vs_1, \vs_2) \cdots G(\vs_{N_\tau - 1}, \vs_{N_\tau}) \psiinit(\vs_{N_\tau}), \label{eq:pimc-pi} \\
E_\text{loc}(\vs) &= \sum_{\vs'} H(\vs, \vs') \frac{\psiinit(\vs')}{\psiinit(\vs)}.
\end{align}

The resulting \cref{eq:pimc} has a similar form to the weighted summation in \cref{eq:cl-obs}. If the weight $\varPi(\svs) \ge 0$ for all $\svs$, we can interpret it as a probability distribution, and estimate \cref{eq:pimc} using the Monte Carlo summation in \cref{eq:monte-carlo}. This method requires to sample the set of auxiliary configurations $\svs$, and each sample contains $N (N_\tau + 1)$ scalar variables, where $N$ is the system size. This is an unbiased estimator of $E$ only when $N_\tau \to \infty$, and in practice we usually need a large $N_\tau$ such that $N_\tau \Delta \tau \gg 1$, which leads to much higher computational cost than Monte Carlo methods for the corresponding classical system. The evaluation of \cref{eq:pimc-pi} not only requires  accurately and efficiently computing the propagator $G(\vs, \vs')$, but also the initial wave function $\psi(\vs)$, and a good choice of $\ket{\psi}$ that is close to the ground state can significantly reduce the magnitude of $N_\tau$ needed. In addition, we note that \cref{eq:pimc-commute} holds not only when estimating the energy, but also any observable that commutes with $\hat{H}$.

This method is commonly referred to as the path integral Monte Carlo~\cite{barker1979quantum}, and a particular method to sample $\svs$ is known as the reptation Monte Carlo~\cite{baroni1999reptation}. For some physical systems, this kind of QMC computations have successfully achieved numerically exact results~\cite{todo2001cluster}, which can serve as the ground truth to test the correctness of other methods even if the ED result is unavailable.

\section{Sign problem}

The above discussion reveals a peculiar caveat in quantum systems: The distribution of a classical ensemble always has $p(\vs) \ge 0$, but the weight in \cref{eq:pimc} may not satisfy $\varPi(\svs) \ge 0$, and the Monte Carlo estimator is invalid in this case, which is known as the sign problem~\cite{loh1990sign, troyer2005computational}. In spin systems, the sign problem is usually a result of the frustrated interactions between the particles~\cite{henelius2000sign}. It is even more prevalent in fermionic systems, where the wave function frequently changes its sign because of the anti-commutation of the particles.

A quick remedy is to factor out the sign from the weight:
\begin{align}
E &= \lim_{N_\tau \to \infty} \sum_\svs \varPi'(\svs) E'_\text{loc}(\svs), \label{eq:pimc-sign} \\
\varPi'(\svs) &= \frac{|\tilde{\varPi}(\svs)|}{\sum_\svs' |\tilde{\varPi}(\svs')|}, \\
E'_\text{loc}(\svs) &= \sign\left( \tilde{\varPi}(\svs) \right) E_\text{loc}(\vs_{N_\tau}). \label{eq:pimc-eloc-sign}
\end{align}
However, the intrinsic complexity of the wave function does not simply disappear. Unlike the classical local free energy $F_\text{loc}(\vs)$ in \cref{eq:fq-loc}, which is usually far below zero when $s$ is around the ground state, the local energy $E'_\text{loc}(\svs)$ in \cref{eq:pimc-eloc-sign} can frequently change its sign as the integration path $\svs$ changes. In \cref{eq:pimc-sign}, there can be many positive and negative terms cancelling out, which leads to a small mean value and a large variance of the estimator, and prevents the result from achieving high accuracy. This problem can also occur for the $E_\text{loc}(\vs_{N_\tau})$ in \cref{eq:pimc}, even if the weights are non-negative.

As a well-studied case, the Hamiltonian is guaranteed to be free of the sign problem if it is stoquastic~\cite{bravyi2008complexity}, i.e., all its off-diagonal entries are non-positive. For example, the ferromagnetic (FM) Heisenberg model on a 1D chain is stoquastic, while the antiferromagnetic (AFM) one is not. Fortunately, the sign problem of a Hamiltonian can be removed with a change of basis in some cases. In the previous example, an AFM Heisenberg model on a non-frustrated graph can be converted to an FM one by applying the Marshall sign rule~\cite{marshall1955antiferromagnetism}. However, such a transformation is unavailable if the sign problem comes from the intrinsic complexity of the wave function, rather than the apparent choice of the basis. Various other schemes of QMC have been proposed to alleviate the sign problem, such as the diffusion Monte Carlo (DMC)~\cite{reynolds1982fixed} and the auxiliary-field quantum Monte Carlo (AFQMC)~\cite{zhang2003quantum}, which are beyond the scope of this thesis. In the next chapter, we present a detailed review of another QMC scheme, namely variational Monte Carlo (VMC), upon which this thesis mainly develops.
