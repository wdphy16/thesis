\chapter{Variational Monte Carlo (VMC)}

The path integral Monte Carlo method in the previous chapter allows us to obtain an unbiased stochastic approximation of the ground state properties for quantum many-body problems, in the same manner as the MCMC method in \cref{ch:mcmc}. Apart from that, another QMC scheme has been proposed to obtain a variational approximation of the target system, which bears the name variational Monte Carlo (VMC)~\cite{scherer2017computational} and resembles the variational method in \cref{ch:cl-var}.

\section{Variational energy}

The ground state of a quantum system in \cref{eq:gs} is already defined by the variational principle, and we only need to rewrite it into a form suitable for Monte Carlo summation. Let $\psi(\vs)$ be the variational ansatz, which is not necessarily normalized, then its energy can be written as
\begin{align}
E_\psi &= \frac{\ev{\hat{H}}{\psi}}{\ip{\psi}} \\
&= \frac
{\sum_{\vs, \vs'} \ip{\psi}{\vs} \mel{\vs}{\hat{H}}{\vs'} \ip{\vs'}{\psi}}
{\ip{\psi}} \\
&= \frac
{\sum_\vs \ip{\psi}{\vs} \ip{\vs}{\psi} \sum_{\vs'} \mel{\vs}{\hat{H}}{\vs'} \frac{\ip{\vs'}{\psi}}{\ip{\vs}{\psi}}}
{\ip{\psi}} \\
&= \sum_\vs q(\vs) E_\text{loc}(\vs), \label{eq:vmc}
\end{align}
where
\begin{align}
q(\vs) &= \frac{|\psi(\vs)|^2}{\ip{\psi}}, \\
E_\text{loc}(\vs) &= \sum_{\vs'} H(\vs, \vs') \frac{\psi(\vs')}{\psi(\vs)}.
\end{align}

The variational energy in \cref{eq:vmc} has a similar form to the variational free energy in \cref{eq:fq}. In the same manner discussed there, we can generate samples of $\vs$ from $q(\vs)$, estimate $E_\psi$ using the Monte Carlo summation, then minimize it w.r.t.\ the parameters in $\psi(\vs)$ using gradient-based optimizers, and a lower $E_\psi$ indicates a better ansatz to approximate of the ground state. Unlike the weight $\varPi(\svs)$ for the path integral Monte Carlo in \cref{eq:pimc}, here we always have $q(\vs) \ge 0$, although the sign problem can still occur in $E_\text{loc}(\vs)$.

\section{Variance of energy}

The variance of the energy $\Var E_\psi = \ev{\hat{H}^2} - \ev{\hat{H}}^2$ under the state $\ket{\psi}$ can also be estimated by Monte Carlo sampling, where the first term can be written as
\begin{align}
\frac{\ev{\hat{H}^2}{\psi}}{\ip{\psi}}
&= \frac
{\sum_{\vs, \vs', \vs''} \ip{\psi}{\vs'} \mel{\vs'}{\hat{H}}{\vs} \mel{\vs}{\hat{H}}{\vs''} \ip{\vs''}{\psi}}
{\ip{\psi}} \label{eq:vmc-var-1} \\
&= \frac
{\sum_\vs \ip{\psi}{\vs} \ip{\vs}{\psi}
\left( \sum_{\vs'} \frac{\ip{\psi}{\vs'}}{\ip{\psi}{\vs}} \mel{\vs'}{\hat{H}}{\vs} \right)
\left( \sum_{\vs''} \mel{\vs}{\hat{H}}{\vs''} \frac{\ip{\vs''}{\psi}}{\ip{\vs}{\psi}} \right)}
{\ip{\psi}} \label{eq:vmc-var-2} \\
\shortintertext{(Assuming $\ip{\vs}{\psi} \neq 0$ for all $\vs$)}
&= \sum_\vs q(\vs) |E_\text{loc}(\vs)|^2. \label{eq:vmc-var}
\end{align}
therefore, the variance of the energy happens to be the same as the variance of the local energy in \cref{eq:vmc}. This variance has the noteworthy property that it reaches zero when the variational state $\ket{\psi}$ reaches the ground state $\ket{\psi_0}$, which is useful in the optimization of $\ket{\psi}$.
For a simple example, in a two-state system $\ket{\psi} = \sqrt{\lambda} \ket{\psi_0} + \sqrt{1 - \lambda} \ket{\psi_1}$, where $\lambda \in [0, 1]$ is a tunable parameter, we have
\begin{equation}
\Var E_\psi = \lambda (1 - \lambda) \Delta E^2, \label{eq:var-two-states}
\end{equation}
where $\Delta E = E_1 - E_0$ is the energy gap, and we can see $\Var E_\psi = 0$ when $\lambda = 1$. However, the variance is also zero if $\ket{\psi}$ is trapped into any excited state, as we can see $\Var E_\psi = 0$ when $\lambda = 0$ in \cref{eq:var-two-states}. This will be discussed in more depth in \cref{ch:varbench}.

Caution should be taken that we assume $\ip{\vs}{\psi} \neq 0$ for all $\vs$ when inserting $\ip{\vs}{\psi}$ into the denominators in \cref{eq:vmc-var-2}. If this assumption is unfulfilled, then the summation in \cref{eq:vmc-var} will only contain the terms with $\ip{\vs}{\psi} \neq 0$, and the variance of the local energy will be a biased estimator of the true variance, as pointed out in~\cite{sinibaldi2023unbiasing}. In practice, when some $\ip{\vs}{\psi}$ are exponentially suppressed and numerically close to zero, they may cause high variance of the variance estimator, which means that the estimated variance and therefore the convergence of the optimization is unreliable. This caveat is similar to the problem of mode collapse in \cref{sec:mode-collapse}, although we are using the variational rather than the unbiased approximation in the quantum case. It does not occur when estimating the energy, because we have $\lim_{\ip{\vs}{\psi} \to 0} q(\vs) E_\text{loc}(\vs) = 0$, while it occurs in the variance as $\ip{\vs}{\psi}$ cancels out in the numerator and the denominator in $q(\vs) |E_\text{loc}(\vs)|^2$.

\section{Complex gradient}

Similar to \cref{eq:fq-grad}, the gradient of $E_\psi$ also needs to be derived in a form suitable for Monte Carlo summation. In the general case where the Hamiltonian $\hat{H}$, the wave function $\psi(\vs)$, and the parameters $\theta$ can be complex-valued, we have
\begin{equation}
\frac{\partial E_\psi}{\partial \theta} = \sum_\vs q(\vs) \left( \left( E_\text{loc}(\vs) - E_\psi \right)^* \frac{\partial \ln \psi(\vs)}{\partial \theta} + \left( E_\text{loc}(\vs) - E_\psi \right) \frac{\partial \ln \psi^*(\vs)}{\partial \theta} \right), \label{eq:vmc-grad-cmpl}
\end{equation}
whose derivation is presented in \cref{append:vmc-grad}. In the simple case where $\hat{H}$, $\psi(\vs)$, and $\theta$ are all real-valued, it simplifies to
\begin{align}
\frac{\partial E_\psi}{\partial \theta} &= 2 \sum_\vs q(\vs) \left( E_\text{loc}(\vs) - E_\psi \right) D_\text{loc}(\vs), \label{eq:vmc-grad-real} \\
D_\text{loc}(\vs) &= \frac{\partial \ln \psi(\vs)}{\partial \theta}.
\end{align}
To reduce the variance when estimating the gradient using Monte Carlo sampling, in the same manner as \cref{eq:fq-grad-baseline}, we shift $D_\text{loc}(\vs)$ to have zero mean, without changing the expectation of the gradient:
\begin{align}
\frac{\partial E_\psi}{\partial \theta} &= 2 \sum_\vs q(\vs) \left( E_\text{loc}(\vs) - E_\psi \right) \left( D_\text{loc}(\vs) - D \right), \label{eq:vmc-grad-baseline} \\
D &= \sum_\vs q(\vs) D_\text{loc}(\vs).
\end{align}

A peculiar nature of quantum systems is that $\hat{H}$, $\psi(\vs)$, and $\theta$ can actually be complex-valued, which has caused substantial confusion because there is no universally accepted and applicable definition for the gradient of a complex-valued function w.r.t.\ complex parameters. For example, the usual definition in complex analysis requires $\lim_{\Delta z \to 0} \frac{f(z + \Delta z) - f(z)}{\Delta z}$ exists for all directions of $\Delta z$ on the complex plane~\cite{rudin1986real}, which is usually unfulfilled for complicated variational ansatzes such as neural networks~\cite{bassey2021survey}. Even if this condition is fulfilled, it usually implies that $f(z)$ is holomorphic and has singular points, which  interfere with the numerical stability of computations.

Fortunately, the variational energy $E_\psi$ is guaranteed to be real when the summation in \cref{eq:vmc} is performed exactly, which allows us to define a ``split'' gradient for the purpose of the gradient descent (GD) optimization in \cref{eq:gd}. Assuming $\theta = \thetar + \rmi \thetai$,  we compute the gradient w.r.t.\ the real parameters $\thetar$ and $\thetai$ respectively, then combine them by
\begin{equation}
\frac{\partial E_\psi}{\partial \theta} = \frac{\partial E_\psi}{\partial \thetar} + \rmi \frac{\partial E_\psi}{\partial \thetai},
\end{equation}
which yields the correct direction of optimization when substituted into \cref{eq:gd}. Using this definition for the gradient in \cref{eq:vmc-grad-cmpl}, we obtain
\begin{align}
\frac{\partial E_\psi}{\partial \theta}
&= \sum_\vs q(\vs) \bigg(
\left( E_\text{loc}(\vs) - E_\psi \right)^* \ptri \ln \psi(\vs) \nonumber \\
&\phantom{{}={}} \qquad\quad + \left( E_\text{loc}(\vs) - E_\psi \right) \ptri \ln \psi^*(\vs)
\bigg).
\end{align}
Because $\frac{\partial f^*(x)}{\partial x} = \left( \frac{\partial f(x)}{\partial x} \right)^*$ when $x$ is real, we have
\begin{align}
\frac{\partial E_\psi}{\partial \theta}
&= \phantom{+ \rmi}\!\sum_\vs q(\vs) \left(
\left( E_\text{loc}(\vs) - E_\psi \right)^* \frac{\partial \ln \psi(\vs)}{\partial \thetar}
+ \left( E_\text{loc}(\vs) - E_\psi \right) \left( \frac{\partial \ln \psi(\vs)}{\partial \thetar} \right)^*
\right) \nonumber \\
&\phantom{=} + \rmi \sum_\vs q(\vs) \left(
\left( E_\text{loc}(\vs) - E_\psi \right)^* \frac{\partial \ln \psi(\vs)}{\partial \thetai}
+ \left( E_\text{loc}(\vs) - E_\psi \right) \left( \frac{\partial \ln \psi(\vs)}{\partial \thetai} \right)^*
\right) \\
% &= \phantom{{}+ \rmi \cdot{}} 2 \Re \sum_\vs q(\vs)
% \left( E_\text{loc}(\vs) - E_\psi \right)
% \left( \frac{\partial \ln \psi(\vs)}{\partial \thetar} \right)^* \nonumber \\
% &\phantom{{}={}} + \rmi \cdot 2 \Re \sum_\vs q(\vs)
% \left( E_\text{loc}(\vs) - E_\psi \right)
% \left( \frac{\partial \ln \psi(\vs)}{\partial \thetai} \right)^*, \\
&= 2 \sum_\vs q(\vs) \left(
\Re\left( \left( E_\text{loc}(\vs) - E_\psi \right) \left( \frac{\partial \ln \psi(\vs)}{\partial \thetar} \right)^* \right)
+ \rmi \Re\left( \left( E_\text{loc}(\vs) - E_\psi \right) \left( \frac{\partial \ln \psi(\vs)}{\partial \thetai} \right)^* \right)
\right). \label{eq:vmc-grad-split}
\end{align}
In addition, we can reduce its variance in the same manner as \cref{eq:vmc-grad-baseline}.

The same gradient, except differing by an overall coefficient of $2$ because of the convention, can be derived from the viewpoint of the Wirtinger calculus~\cite{wirtinger1927formalen}, also known as the $\bbC \bbR$-calculus~\cite{kreutz2009complex}. In this theory, any complex-valued function can be written as $f(z, z^*)$, which is holomorphic in $z$ and $z^*$ respectively, and we can take the gradient $\frac{\partial f}{\partial z}$ and the co-gradient $\frac{\partial f}{\partial z^*}$ while treating the other variable as constant. It has been proven that
\begin{equation}
\frac{\partial f}{\partial z} = \frac{1}{2} \left( \frac{\partial f}{\partial x} - \rmi \frac{\partial f}{\partial y} \right), \quad
\frac{\partial f}{\partial z^*} = \frac{1}{2} \left( \frac{\partial f}{\partial x} + \rmi \frac{\partial f}{\partial y} \right),
\end{equation}
where $z = x + \rmi y$. For the purpose of GD, we substitute the co-gradient $\frac{\partial E_\psi}{\partial \theta^*}$ into \cref{eq:gd}, which is the same as \cref{eq:vmc-grad-split} except without the coefficient of $2$. Automatic differentiation (AD) software can use the Wirtinger calculus with the chain rule of derivatives to compute $\frac{\partial \ln \psi}{\partial \theta^*}$~\cite{kramer2024tutorial}, which usually takes less computation time than computing $\frac{\partial \ln \psi}{\partial \thetar}$ and $\frac{\partial \ln \psi}{\partial \thetai}$ separately, but care should be taken that different software can have different conventions for the sign of the imaginary part and the overall coefficient.

It is worth discussing the $\Re$ notation in \cref{eq:vmc-grad-split}. The gradients $\frac{\partial E_\psi}{\partial \thetar}$ and $\frac{\partial E_\psi}{\partial \thetai}$ are both real when evaluated exactly, because they are gradients of a real-valued function w.r.t.\ real parameters. Therefore, the sums $\sum_\vs q(\vs) \left( E_\text{loc}(\vs) - E_\psi \right) \left( \frac{\partial \ln \psi(\vs)}{\partial \thetar} \right)^*$ and $\sum_\vs q(\vs) \left( E_\text{loc}(\vs) - E_\psi \right) \left( \frac{\partial \ln \psi(\vs)}{\partial \thetai} \right)^*$ are also real when evaluated exactly. However, when using Monte Carlo sampling, the estimators $\bbE_\text{MC}\left[ \left( E_\text{loc} - E_\psi \right) \left( \frac{\partial \ln \psi}{\partial \thetar} \right)^* \right]$ and $\bbE_\text{MC}\left[ \left( E_\text{loc} - E_\psi \right) \left( \frac{\partial \ln \psi}{\partial \thetai} \right)^* \right]$ can have imaginary parts because of the discrepancy of samples, which are discarded in \cref{eq:vmc-grad-split} before outputting the gradient and updating the parameters in \cref{eq:gd}.

In some implementations of VMC, as well as stochastic gradient descent (SGD) in other complex-valued optimization problems, the $\Re$ notation is ignored either deliberately or unintentionally, and the imaginary parts of the estimated $\frac{\partial E_\psi}{\partial \thetar}$ and $\frac{\partial E_\psi}{\partial \thetai}$ are mixed into the updates to $\thetai$ and $\thetar$ respectively. This additional noise in the gradient may affect the convergence of SGD, as discussed in \cref{sec:gd}. To the author's knowledge, there is no strong evidence that either implementation consistently outperforms the other. However, a mistake that occasionally occurs is to take the real parts before multiplying, i.e., $\Re\left( E_\text{loc} - E_\psi \right) \Re\left( \frac{\partial \ln \psi}{\partial \thetar} \right)$, which changes the result even if exactly performing the summation over $\vs$. We refer to Ref.~\cite{bassey2021survey} as a recent survey on complex-valued neural networks, including their optimization, in various fields of machine learning.

\section{Amplitude and phase of quantum ansatz}

To avoid the aforementioned intricacy of complex-valued functions,







\section{Stochastic reconfiguration (SR)}
\label{sec:sr}

\footnote{The name ``stochastic reconfiguration'' is also used in diffusion Monte Carlo (DMC), where it refers to the technique to reconfigure the number of walkers~\cite{assaraf2000diffusion}. Both the names originate from the earlier work on Green's function Monte Carlo~\cite{sorella1998green}.}

QMC over VMC
