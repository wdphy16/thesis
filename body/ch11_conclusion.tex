\chapter{Conclusion}
\label{ch:conclusion}

\section{Main results}

In this thesis, we have applied exact sampling with ARNN to both classical and quantum many-body systems. For classical systems, in \cref{sec:made} we have shown that ARNN obtains substantially more accurate variational free energy than the traditional NMF and Bethe ansatzes on the square Ising model, estimates the residual entropy for the triangular Ising model, and retrieves the memorized patterns for the Hopfield model. It also outperforms NMF and Bethe ansatzes on the SK model and the inverse SK problem, showing its potential to capture exponentially many modes of the complicated many-body distribution in polynomial amount of computation. The globally interacting systems only require small dense ARNNs with $O(N^2)$ parameters, while larger ARNNs have better accuracies for the Ising models on the lattices. The convolutional ARNN has better parameter efficiency than the dense one on the lattices because of the incorporated translational symmetry and locality of the physical system, and it needs a sufficiently large receptive field to capture the long-range correlations around the critical point.

In \cref{sec:twobo} we have presented the TwoBo ansatz, which incorporates the knowledge of the Boltzmann distribution and the sparse two-body interacting Hamiltonian into the neural network design. When applied to the EA model with binary interactions on 2D grids, 3D grids, and RRGs, it achieves more accurate variational free energy and ground state energy, substantially faster convergence in training, and polynomially higher parameter efficiency on the regular grids, compared to the dense ARNN and the previously proposed RNN.

In \cref{sec:arnn-mcmc} we have demonstrated the NCUS sampling method, which consists of the cluster updates constructed by the AR property of the ansatz and the symmetry operations of the physical system. When applied to the Ising model with the second-order phase transition and the FPM with the first-order phase transition, it produces autocorrelation times lower than the conventional global update method using the same ARNN, and comparable to the Wolff cluster update method specially designed for the Ising model. This advantage is particularly substantial around the critical point, leading to more reliable estimations of observables with lower error bars.

For quantum systems, in \cref{ch:tensor-rnn} we have developed the tensor-RNN ansatz with both advantages of TN and RNN. Compared to MPS which can only produce $O(\ln \chi)$ entanglement entropy and exponentially decaying correlations, there is both analytical and numerical evidence for tensor-RNN to produce the 2D area law of entanglement entropy, as well as numerical evidence for the algebraically decaying correlations in 2D critical systems. Compared to PEPS which has exponentially high computational cost, tensor-RNN supports efficient evaluation of the wave function in polynomial time, as well as exact sampling. As the result, it has higher parameter efficiency than MPS, and also achieves better accuracy of variational approximation than PEPS in a practical computation time. On the other hand, compared to conventional RNN architectures which are difficult to analytically study and adaptively scale up, the physical properties of tensor-RNN have been analytically investigated, and its accuracy can be systematically improved by increasing the bond dimension. The exact mapping from MPS to tensor-RNN leads to the hierarchical initialization, which makes it possible to optimize tensor-RNN with numerous parameters.

Lastly, in \cref{ch:varbench} we have presented the VarBench project, which is an extensive collection of benchmarks for many kinds of variational methods and quantum many-body systems in more than $400$ numerical studies. The metric used in these benchmarks is the V-score, a universal quantity to measure the accuracy of any variational approximation, as well as the hardness of simulating any Hamiltonian, which is derived from the relation between the variance and the accuracy of the variational energy. Using this metric, we have identified certain hard Hamiltonians that are not yet solved by any available method, which can be the targets to demonstrate the quantum advantage in the future.

\section{Outlook}

A large topic that has been deliberately omitted in this thesis is the symmetries in the ansatz. Although we can always implement the symmetries ``out'' of the ansatz using a mixture of symmetrized copies of the ansatz, commonly known as the quantum number projection~\cite{tahara2008variational}, it introduces a factor of $|G|$ on the computation time, where $|G|$ is the size of the symmetry group. The more feasible approach is to implement the symmetries ``in'' the architecture of the ansatz, like CNNs and group equivariant neural networks, and even the $\mathrm{SU}(2)$ ClebschTree~\cite{vieijra2021many}. However, a common criticism of ARNN is that it breaks the symmetries as long as it artificially defines an AR order. It can be promising to implement ARNNs with hierarchical architectures and in frequency or wavelet spaces~\cite{nash2021generating, bialas2022hierarchical, hoogeboom2022autoregressive, mattar2024wavelets}, especially considering the recent theory that the diffusion models, being the state-of-the-art models for 2D images, can be formulated as AR models in the frequency space. Meanwhile, we mention that the constraint of fixed total magnetization or fixed number of fermions can be implemented in ARNN as described in Appendix D.2 of Ref.~\cite{hibat2020recurrent}.

Another issue concerning the title of this thesis is whether we can find clear evidence that exact sampling outperforms MCMC sampling in real use cases of variational optimization. Although we have been using ARNN intuitively with AR sampling since the early studies, there are results showing that ARNN with AR sampling does not significantly outperform the same ARNN with MCMC sampling in terms of accuracy and computation time. Ref.~\cite{bukov2021learning} even provides evidence that the full-basis simulation does not produce substantially more accurate variational energy than the Monte Carlo sampling for a small quantum system. Moreover, despite the recent achievement of ViT for the modeling of natural images, it seems that the autoregressive architecture of ViT is unnecessary for physical systems~\cite{viteritti2023transformer, rende2024queries}. Although we can show the advantage of exact sampling in toy models such as \cref{fig:mag-hist}, it is unclear how badly the discrepancy between the ansatz distribution and its MCMC samples affects the variational optimization, and there have been studies of the convergence properties of SGD with autocorrelated samples from Markov chains~\cite{sun2018markov}. A future step of research is to show the advantage of exact sampling in controlled experiments on systems with more complicated energy landscapes.

% Optimization

Speaking of the ViT, it motivates us to introduce more techniques that are proven effective in machine learning practices into the solution of physical systems. A promising technique is the positional encoding~\cite{ke2021rethinking}, which allows convolutional ARNNs to know its position in the system. Methods to scale the positional encoding have been developed to enable longer context windows in language models~\cite{liu2024scaling, peng2024yarn}, which may be helpful for scaling physical systems to the thermodynamic limit.
