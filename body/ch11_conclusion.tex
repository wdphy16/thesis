\chapter{Conclusion}
\label{ch:conclusion}

\section{Main results}

In this thesis, we have applied exact sampling with ARNN to both classical and quantum many-body systems. For classical systems, in \cref{sec:made} we have shown that ARNN obtains substantially more accurate variational free energy than the traditional NMF and Bethe ansatzes on the square Ising model, estimates the residual entropy for the triangular Ising model, and retrieves the memorized patterns for the Hopfield model. It also outperforms NMF and Bethe ansatzes on the SK model and the inverse SK problem, showing its potential to capture exponentially many modes of the complicated many-body distribution in polynomial amount of computation. The globally interacting systems only require small dense ARNNs with $O(N^2)$ parameters, while larger ARNNs have better accuracies for the Ising models on the lattices. The convolutional ARNN has better parameter efficiency than the dense one on the lattices because of the incorporated translational symmetry and locality of the physical system, and it needs a sufficiently large receptive field to capture the long-range correlations around the critical point.

In \cref{sec:twobo} we have presented the TwoBo ansatz incorporating the knowledge of the Boltzmann distribution and the sparse two-body interacting Hamiltonian into the neural network design. When applied to the EA model with binary interactions on 2D grids, 3D grids, and RRGs, it achieves more accurate variational free energy and ground state energy, substantially faster convergence in training, and polynomially higher parameter efficiency on the regular grids, compared to the dense ARNN and the previously proposed RNN.

In \cref{sec:arnn-mcmc} we have demonstrated the NCUS sampling method utilizing the cluster updates constructed by the AR property of the ansatz and the symmetries of the physical system. When applied to the Ising model with the second-order phase transition and the FPM with the first-order phase transition, it produces lower autocorrelation times than the conventional global update method, which is comparable to the Wolff algorithm specially designed for the Ising model. This advantage is particularly substantial around the critical point, leading to more reliable estimations of observables with lower error bars.

For quantum systems, in \cref{ch:tensor-rnn} we have developed the tensor-RNN ansatz with both advantages of TN and RNN. Compared to MPS which can only produce $O(\ln \chi)$ entanglement entropy and exponentially decaying correlations, there are both analytical and numerical evidence for tensor-RNN to produce the 2D area law of entanglement entropy, as well as numerical evidence for the algebraically decaying correlations in 2D critical systems. Compared to PEPS which has exponentially high computational cost, tensor-RNN supports efficient evaluation of the wave function in polynomial time, as well as exact sampling. As a result, tensor-RNN has higher parameter efficiency than MPS, and also achieves better accuracy of variational approximation than PEPS in a practical computation time. On the other hand, compared to conventional RNN architectures which are difficult to analytically study and adaptively scale up, the physical properties of tensor-RNN have been analytically investigated, and its accuracy can be systematically improved by increasing the bond dimension. The exact mapping from MPS to tensor-RNN leads to the hierarchical initialization, which makes it possible to optimize tensor-RNN with numerous parameters.

Lastly, in \cref{ch:varbench} we have presented the VarBench project, which is an extensive collection of benchmarks for many kinds of variational methods and quantum many-body systems in more than $400$ numerical studies. The metric used in these benchmarks is the V-score, a universal quantity to measure the accuracy of any variational approximation, as well as the hardness of simulating any Hamiltonian, which is derived from the relation between the variance and the accuracy of the variational energy. Using this metric, we have identified certain hard Hamiltonians that are not yet solved by any available method, which can be the targets to demonstrate the quantum advantage in the future.

\section{Outlook}

Is AR sampling useful

symmetry

optimization

more ML-like approach, positional encoding
