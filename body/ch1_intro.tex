\chapter{Background}

\section{Many-body systems}

\section{Computational methods}

\section{Machine learning}

\section{Outline}

This thesis is written in a mostly self-contained way, assuming the reader has the basic backgrounds of statistics and quantum mechanics. It is organized as follows:
\begin{itemize}
\item \Cref{ch:systems} provides the definitions of the classical and quantum many-body systems studied in this thesis, including the classical Ising model and its variants, the quantum Heisenberg model, and the transverse-field Ising model (TFIM).
It discusses the brief histories and the motivations to study them, and the issues of geometrical frustration and random interactions that create disordered states of the systems. These systems exhibit rich and exotic physical phenomena, which pose challenge to the computational studies we will present.
\item \Cref{ch:mcmc} starts the discussion of computational studies on classical many-body systems. It reviews the Markov chain Monte Carlo (MCMC) method, which is the most traditional and widely-used method to stochastically estimate the observables under the complicated many-body distributions of physical systems.
It discusses the issues of autocorrelation time, critical slowing down, burn-in stage before equilibrium, and mode collapse, which will reoccur in all other methods based on MCMC, as well as in quantum systems.
Before introducing the Markov chain sampling, it also defines the concept of exact sampling with simple examples, which will be the focus of this thesis.
\item \Cref{ch:cl-var} reviews the variational method to approximate the Boltzmann distributions of classical many-body systems.
It gives detailed explanations on the optimization methods and the gradient estimator for the variational free energy, as well as a comparison to MCMC.
Then it lists some commonly used variational ansatzes, including the naive mean-field ansatz, the Bethe ansatz, and neural network ansatzes, with a self-contained formulation of neural networks.
This variational method has close resemblance to the variational Monte Carlo method for quantum
systems in \cref{ch:vmc}.
\item \Cref{ch:arnn} introduces the autoregressive neural networks (ARNNs). Using the power of neural networks, they enable practical applications of exact sampling for complicated many-body distributions.
It presents the basic architectures of ARNN from Ref.~\cite{wu2019solving}, the sparse two-body ARNN (TwoBo) from Ref.~\cite{biazzo2024sparse} to substantially improve the efficiency using the sparsity structure of the physical system, and the neural cluster updates with symmetries (NCUS) from Ref.~\cite{wu2021unbiased} to remove the bias from the variational approximation, with numerical results demonstrating their accuracy and efficiency.
\item \Cref{ch:qmc} moves the discussion to computational studies on quantum many-body systems. It reviews the unbiased methods to evaluate the ground state, including the theoretical formulation of imaginary time evolution (ITE), the exact diagonalization (ED) method derived from the former, and the quantum Monte Carlo (QMC) methods, in particular the path integral Monte Carlo (PIMC).
It discusses the impact of the energy gap size on the convergence speeds of these methods, and the notorious sign problem, which will reoccur in \cref{ch:vmc}.
The results of ED and QMC are used as baselines for other numerical studies, including the benchmarks in \cref{ch:varbench}.
\item \Cref{ch:vmc} reviews the variational Monte Carlo (VMC) method, which is more widely-used in quantum problems than the variational method in classical problems, because VMC has higher efficiency than the unbiased QMC schemes.
It skips the formulation similar to the classical variational method, and mentions the relation between the variance and the accuracy of the variational energy, which will be the fundament of the benchmarks in \cref{ch:varbench}.
It emphasizes the particular difficulty of complex gradients for wave functions, and presents the stochastic reconfiguration (SR) method to help the optimization in complicated energy landscapes of quantum systems.
Then it lists some commonly used variational ansatzes, including the Jastrow ansatz, the Gutzwiller projected states, and the neural quantum states (NQSs), with a discussion on parameterizing and optimizing the phase of the wave function.
\item \Cref{ch:tn} reviews the tensor network (TN) method, which is an alternative to VMC for the variational approximation of quantum states. TNs have well-understood physical properties, such as entanglement entropy and spatial correlations, as well as the advantage of exact contraction if the computational cost allows.
In particular, it presents the matrix product state (MPS) and the matrix product operator (MPO) architectures in 1D, as well as the density matrix renormalization group (DMRG) algorithm to optimize MPS, with brief introductions to architectures in 2D and higher.
\item \Cref{ch:tensor-rnn} introduces the tensor-RNN from Ref.~\cite{wu2023tensor}, a variational ansatz  combining the strengths of TN in \cref{ch:tn} and NQS in \cref{ch:vmc}. This ansatz supports exact sampling and efficient evaluation, captures the desired analytical properties of entanglement entropy and spatial correlations in 2D, and produces systematically improved accuracy as the computational cost increases. These advantages are supported by both theoretical and numerical evidences.
\item \Cref{ch:varbench} introduces the VarBench project from Ref.~\cite{wu2023variational}. It first proposes the V-score, a universal metric based on the variance of variational energy, which measures the accuracy of any variational approximation, as well as the hardness of simulating any Hamiltonian. This score is used in an extensive collection of benchmarks for many kinds of variational methods and quantum many-body systems, which identifies certain hard Hamiltonians that are not yet well solved, and can be the targets to demonstrate the quantum advantage.
\item \Cref{ch:conclusion} concludes the thesis, summarizes the main results that improve the accuracy and the efficiency of computational studies in both classical and quantum many-body systems, and poses directions and open questions for further research.
\end{itemize}
