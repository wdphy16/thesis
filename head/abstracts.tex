% English abstract
\cleardoublepage
\chapter*{Abstract}
\markboth{Abstract}{Abstract}
\addcontentsline{toc}{chapter}{Abstract (English/Français)} % adds an entry to the table of contents

Many-body systems at low temperature have revealed non-trivial phases of materials, such as spin liquids, which have found applications in the evolving fields of superconductivity, nanoelectronics, and quantum computing.
Their exponentially large state spaces generally forbid us to obtain their exact solutions, and motivate us to develop sampling methods instead.
The Markov chain Monte Carlo (MCMC) method is the traditional way to sample from complicated many-body distributions, but it suffers from issues such as autocorrelation time and critical slowing down.
Exact sampling methods with variational ansatzes are preferred to overcome these issues, but it had been difficult to construct ansatzes that are both expressive enough to capture the rich physical phenomena and efficient enough to sample from.

The recent development of neural networks sheds new light on this area.
In this thesis, we present autoregressive neural networks (ARNNs) as a powerful family of variational ansatzes, which share the same framework as the modern large language models.
They can be modularly constructed with higher expressiveness than previous ansatzes, while support exact sampling and efficient evaluation.

For classical systems, we present the sparse two-body (TwoBo) ansatz to incorporate the sparsity of the physical system in ARNN.
It substantially improves the computational efficiency and the convergence of variational optimization compared to the conventional dense ARNN, without loss of expressiveness.
Then we present the sampling method of neural cluster updates with symmetries (NCUS), which finds a balance between the local updates of MCMC and the global updates of conventional exact sampling methods.
It removes the bias in variational approximation and substantially reduces the autocorrelation time.

For quantum systems, we present the tensor-RNN ansatz, which combines the strengths of tensor networks (TNs) and recurrent neural networks (RNNs).
It captures the desired analytical properties of entanglement entropy and spatial correlations in two-dimensional systems, and produces systematically improved accuracy as the computational budget increases.
These advantages are supported by both theoretical and numerical evidence.
Lastly, we present the results of VarBench, an extensive project to benchmark the performances of variational methods on quantum many-body systems, which witnesses the development of this field and identifies certain targets for the demonstration of quantum advantage in the future.
The metric used in these benchmarks is the V-score, a universal quantity to measure the accuracy of any variational approximation, as well as the hardness of simulating any Hamiltonian.
These techniques together enable more accurate approximation of systems with larger size and higher complexity.

% French abstract
\begin{otherlanguage}{french}
\cleardoublepage
\chapter*{Résumé}
\markboth{Résumé}{Résumé}

Les systèmes à plusieurs corps à basse température ont révélé des phases non triviales de matériaux, tels que les liquides de spin, qui ont trouvé des applications dans les domaines en évolution de la supraconductivité, de la nanoélectronique et de l'informatique quantique.
Leurs espaces d'états exponentiellement grands nous empêchent généralement d'obtenir leurs solutions exactes et nous incitent à développer des méthodes d'échantillonnage à la place.
La méthode de Monte-Carlo par chaîne de Markov (MCMC) est le moyen traditionnel d'échantillonner à partir de distributions compliquées à plusieurs corps, mais elle souffre de problèmes tels que le temps d'autocorrélation et le ralentissement critique.
Les méthodes d'échantillonnage exactes avec des ansatzes variationnels sont préférées pour surmonter ces problèmes, mais il a été difficile de construire des ansatzes qui soient à la fois suffisamment expressifs pour capturer la richesse des phénomènes physiques et suffisamment efficaces pour échantillonner.

Le développement récent des réseaux neuronaux jette une lumière nouvelle sur ce domaine.
Dans cette thèse, nous présentons les réseaux neuronaux autorégressifs (ARNNs) comme une famille puissante d'ansatzes variationnels, qui partagent le même cadre que les grands modèles de langage modernes.
Ils peuvent être construits de manière modulaire avec une plus grande expressivité que les ansatzes précédents, tout en supportant l'échantillonnage exact et l'évaluation efficace.

Pour les systèmes classiques, nous présentons l'ansatz à deux corps clairsemés (TwoBo) pour intégrer la clairsemée du système physique dans l'ARNN.
Il améliore considérablement l'efficacité de calcul et la convergence de l'optimisation variationnelle par rapport à l'ARNN dense conventionnel, sans perte d'expressivité.
Nous présentons ensuite la méthode d'échantillonnage des mises à jour de clusters neuronaux avec symétries (NCUS), qui trouve un équilibre entre les mises à jour locales de MCMC et les mises à jour globales des méthodes d'échantillonnage exactes conventionnelles.
Elle supprime le biais de l'approximation variationnelle et réduit considérablement le temps d'autocorrélation.

Pour les systèmes quantiques, nous présentons l'ansatz tenseur-RNN, qui combine les atouts des réseaux tensoriels (TNs) et des réseaux neuronaux récurrents (RNNs).
Il capture les propriétés analytiques souhaitées de l'entropie d'intrication et des corrélations spatiales dans les systèmes bidimensionnels, et produit une précision systématiquement améliorée à mesure que le budget de calcul augmente.
Ces avantages sont étayés par des évidences théoriques et numériques.
Enfin, nous présentons les résultats de VarBench, un vaste projet visant à évaluer les performances des méthodes variationnelles sur les systèmes quantiques à plusieurs corps, qui témoigne du développement de ce domaine et identifie certaines cibles pour la démonstration de l'avantage quantique à l'avenir.
La métrique utilisée dans ces résultats est le V-score, une quantité universelle pour mesurer l'exactitude de toute approximation variationnelle, ainsi que la difficulté de la simulation de tout hamiltonien.
Ces techniques ensemble permettent des approximations plus exactes de systèmes de plus grande taille et complexité.

\end{otherlanguage}
