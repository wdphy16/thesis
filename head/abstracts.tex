% English abstract
\cleardoublepage
\chapter*{Abstract}
\markboth{Abstract}{Abstract}
\addcontentsline{toc}{chapter}{Abstract (English/Français)} % adds an entry to the table of contents

Many-body systems at low temperature have revealed non-trivial phases of materials, such as spin liquids, which have found applications in the evolving fields of superconductivity, nanoelectronics, and quantum computing.
Their exponentially large state spaces generally forbid us to obtain their exact solutions, and motivate us to develop sampling methods instead.
The Markov chain Monte Carlo (MCMC) method is the traditional way to sample from complicated many-body distributions, but it suffers from issues such as autocorrelation time and critical slowing down.
Exact sampling methods with variational ansatzes are preferred to overcome these issues, but it had been difficult to construct ansatzes that are both expressive enough to capture the rich physical phenomena and efficient enough to sample from.

The recent development of neural networks sheds new light on this area.
In this thesis, we present autoregressive neural networks (ARNNs) as a powerful family of variational ansatzes, which share the same framework as the modern large language models.
They can be modularly constructed with higher expressiveness than previous ansatzes, while support exact sampling and efficient evaluation.

For classical systems, we present the sparse two-body (TwoBo) ansatz to incorporate the sparsity of the physical system in ARNN.
It substantially improves the computational efficiency and the convergence of variational optimization compared to the conventional dense ARNN, without loss of expressiveness.
Then we present the sampling method of neural cluster updates with symmetries (NCUS), which finds a balance between the local updates of MCMC and the global updates of conventional exact sampling methods.
It removes the bias in variational approximation and substantially reduces the autocorrelation time.

For quantum systems, we present the tensor-RNN ansatz, which combines the strengths of tensor networks (TNs) and recurrent neural networks (RNNs).
It captures the desired analytical properties of entanglement entropy and spatial correlations in two-dimensional systems, and produces systematically improved accuracy as the computational budget increases.
These advantages are supported by both theoretical and numerical evidence.
Lastly, we present the results of VarBench, an extensive project to benchmark the performances of variational methods on quantum many-body systems, which witnesses the development of this field and identifies certain targets for the demonstration of quantum advantage in the future.
The metric used in these benchmarks is the V-score, a universal quantity to measure the accuracy of any variational approximation, as well as the hardness of simulating any Hamiltonian.
These techniques together enable more accurate approximation of systems with larger size and higher complexity.

% French abstract
\begin{otherlanguage}{french}
\cleardoublepage
\chapter*{Résumé}
\markboth{Résumé}{Résumé}

À basse température, les systèmes à plusieurs corps ont révélé des phases non triviales de matériaux, tels que les liquides de spin, qui ont trouvé des applications dans les domaines de la supraconductivité, la nanoélectronique et l'informatique quantique, en constante évolution. Leur espace de Hilbert de taille exponentielle nous empêche généralement d'obtenir des solutions exactes et nous incitent en conséquence à développer des méthodes d'échantillonnage. La méthode de Monte Carlo par chaîne de Markov (MCMC) est le moyen traditionnel d'échantillonner des distributions complexes de systèmes à N corps, mais elle est affectée de problèmes tels que le temps d'autocorrélation et le phénomène de ralentissement critique. Les méthodes d'échantillonnage exact appliquées à des ansatzes variationnels sont à privilégier afin de surmonter ces difficultés. Cependant, la conception de tels ansatzes, à la fois suffisamment expressifs pour capturer une phénoménologie riche et à l'échantillonnage suffisamment efficace, est longtemps restée un défi.

Le développement récent des réseaux neuronaux apporte un éclairage nouveau sur ce domaine. Dans cette thèse, nous présentons les réseaux neuronaux autorégressifs (ARNN) comme une famille puissante de modèles variationnels, qui s'insère dans le même cadre que les grands modèles de langage modernes. Ces derniers peuvent être construits de manière modulaire avec une plus grande expressivité que les ansatzes précédemment utilisés, tout en permettant un échantillonnage exact et une évaluation efficace.

Pour les systèmes classiques, nous présentons l'ansatz sparse à deux corps (TwoBo) pour incorporer dans l'ARNN la structure faiblement dense du système physique. Ceci améliore considérablement l'efficacité numérique et la convergence de l'optimisation variationnelle par rapport à l'ARNN dense conventionnel, sans perte d'expressivité. Par la suite, nous présentons la méthode neurale d'échantillonnage par mise à jour de clusters avec symétries (NCUS), qui concilie les mises à jour locales du MCMC et les mises à jour globales des méthodes conventionnelles d'échantillonnage exact. Cela supprime le biais de l'approximation variationnelle et réduit considérablement le temps d'autocorrélation.

Pour les systèmes quantiques, nous présentons l'ansatz tenseur-RNN, qui combine les atouts des réseaux tensoriels (TN) et des réseaux neuronaux récurrents (RNN).
Il capture les propriétés analytiques souhaitées de l'entropie d'enchevêtrement et des corrélations spatiales dans les systèmes bidimensionnels, et produit une précision systématiquement accrue au fur et à mesure que le budget de calcul est augmenté. Ces avantages sont étayés par des preuves théoriques et numériques. Enfin, nous présentons les résultats de VarBench, un vaste projet visant à comparer les performances des méthodes variationnelles sur les systèmes quantiques à multiples corps en interaction et qui témoigne du développement de ce domaine tout en identifiant certains objectifs pour la démonstration de l'avantage quantique à l'avenir. La métrique utilisée dans ces benchmarks est le V-score, une quantité universelle permettant de mesurer la précision de toute approche variationnelle, ainsi que le degré de difficulté de la simulation pour tout hamiltonien. Dans l'ensemble, ces techniques permettent une approximation plus précise de systèmes de plus grande dimension et de plus grande complexité.

\end{otherlanguage}
